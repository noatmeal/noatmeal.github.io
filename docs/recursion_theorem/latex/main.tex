\documentclass{article}

\usepackage[title]{appendix}

\usepackage{url}

\usepackage{amsmath}
\usepackage{amssymb}

% For theorems, definitions, etc.
\usepackage{amsthm}

% For math symbols
\usepackage{amsmath}

\theoremstyle{definition}
\newtheorem{definition}{Definition}[section]
% For autoref so that you don't have to put "thing \ref{label}" every time.
% Just put "\autoref{label}" and it will put the right thing.
\newcommand*{\definitionautorefname}{Definition}

\theoremstyle{definition}
\newtheorem{example}{Example}[section]
\newcommand*{\exampleautorefname}{Example}

\theoremstyle{plain}
\newtheorem{aRule}{Rule}[section]
\newcommand*{\aRuleautorefname}{Rule}

\theoremstyle{remark}
\newtheorem{convention}{Convention}[section]
\newcommand*{\conventionautorefname}{Convention}

\theoremstyle{plain}
\newtheorem{theorem}{Theorem}[section]
% Doesn't need an autorefname decleration because it's already been made.

\theoremstyle{remark}
\newtheorem{remark}{Remark}[section]
\newcommand*{\remarkautorefname}{Remark}

\theoremstyle{plain}
\newtheorem{lemma}{Lemma}[section]
\newcommand*{\lemmaautorefname}{Lemma}

\theoremstyle{plain}
\newtheorem{proposition}{Proposition}[section]
\newcommand*{\propositionautorefname}{Proposition}

\theoremstyle{plain}
\newtheorem{axiom}{Axiom}[section]
\newcommand*{\axiomautorefname}{Axiom}

\theoremstyle{plain}
\newtheorem{corollary}{Corollary}[section]
\newcommand*{\corollaryautorefname}{Corollary}

\AtBeginDocument{\def\chapterautorefname{Chapter}}%
\AtBeginDocument{\def\sectionautorefname{Section}}%
\AtBeginDocument{\def\subsectionautorefname{Section}}%
\AtBeginDocument{\def\subsubsectionautorefname{Section}}%
\AtBeginDocument{\def\paragraphautorefname{Paragraph}}%


% For links to use while reading the pdf on a computer. Also provides nice 
% colors for links.
\usepackage[unicode]{hyperref}
\hypersetup{
    colorlinks=true,
    linkcolor=blue
}
% Puts you at top of figures in hyperlinks
\usepackage[all]{hypcap}

\title{The Recursion Theorem}
\author{\url{https://noatmeal.github.io/}}
\date{}

\begin{document}

\maketitle

A useful item to have in one's toolbox while working in ZFC set 
theory\footnote{Short for Zermelo–Fraenkel set theory. See 
\url{https://en.wikipedia.org/wiki/Zermelo-Fraenkel_set_theory} for more 
information.} is the ability to create inductive definitions. My goal in this 
post is to prove a theorem that allows us to do exactly that whenever the 
proper conditions are present. We'll also work out a couple of fun examples at 
the end.

\section{Preliminaries}

If the following material doesn't make much sense, then a great review 
can be found in the first half of \cite{naiiveSet}. A proof for the main 
theorem of this post can be found in that text as well.

\subsection{Natural Numbers}

\begin{definition}
  A set \( I \) is inductive if and only if \( \emptyset \in I \) and 
  \( \forall y \in I (y \cup \{y\} \in I )\).
\end{definition}

\begin{example}
  Here are some elements one would find in an inductive set
  \begin{itemize}
    \item \( \emptyset \)
    \item \( \emptyset \cup \{\emptyset\} = \{\emptyset\}\)
    \item \( 
            \{\emptyset\} \cup \{\{\emptyset\}\} 
            = \{\emptyset, \{\emptyset\}\}
          \)
    \item \(
            \{\emptyset, \{\emptyset\}\} \cup \{ \{\emptyset, \{\emptyset\}\} \} 
            = 
            \{ \emptyset, \{\emptyset\}, \{\emptyset, \{ \emptyset \} \} \} 
          \)
  \end{itemize}
  \label{example:inductive_set}
\end{example}

\begin{axiom}[Infinity]
  \( \exists I ( I \text{ is inductive})\)
  \label{axiom:infinity}
\end{axiom}

\begin{axiom}[Comprehension]
  \( 
    \forall v \exists y \forall x 
    (x \in y \Leftrightarrow x \in v \land \phi(x)) 
  \) 
  where 
  \( \phi \) is a formula that \( y \) is not free in.
  \label{axiom:comprehension}
\end{axiom}


\begin{definition}
  \label{def:natural_numbers}
  Let \( I \) be a set who's existence is guaranteed by 
  \autoref{axiom:infinity}. We use \autoref{axiom:comprehension} to form the set 
  \[ 
    \mathbb{N} 
    = 
    \{y \in I : \forall J ( J \text{ is inductive} \Rightarrow y \in J )\}
  \] 
  One should think of \( \mathbb{N} \) as the 
  ``smallest" inductive set\footnote{We will prove this in 
  \autoref{thm:third_peano_axiom}}. An 
  element of \( \mathbb{N} \) is called a \textit{natural number}.
\end{definition}

\begin{example}
  \label{example:compact_numbers}
  \autoref{fig:natural_compact_names} shows a few compact names for natural 
  numbers that are often used in place of the set theory 
  notation\footnote{Note that \( 1 \) is considered the first natural number in 
  this post.}.
\end{example}

\begin{figure}[h]
    \centering
    \begin{tabular}{ c | c }
      Element of \(\mathbb{N}\)  & Compact Name  \\
      \hline
      \(\emptyset\)                                                        & \(1\) \\
      \(\{\emptyset\}\)                                                    & \( 2 \) \\ 
      \(\{\emptyset, \{\emptyset\}\}\)                                     & \(3\) \\
      \(\{ \emptyset, \{\emptyset\}, \{\emptyset, \{ \emptyset \} \} \} \) & 4 
    \end{tabular} 
    \caption{Compact names for elements of \( \mathbb{N}\).}
    \label{fig:natural_compact_names}
\end{figure}

\begin{definition}
  The function \( S: \mathbb{N} \rightarrow \mathbb{N} \) defined by 
  \( S(n) = n \cup \{n\} \) for all \( n \in \mathbb{N} \) is called the 
  \textit{successor function on} \( \mathbb{N} \). For \( n \in \mathbb{N} \), 
  we will refer \( S(n) \) as \( n^{+} \).
\end{definition}

\begin{example}
  Recall \autoref{example:inductive_set} and \autoref{example:compact_numbers}. 
  One can deduce that 
  \begin{itemize}
    \item \(1^{+} = 2 \)
    \item \(2^{+} = 3\)
    \item \(3^{+} = 4\)
  \end{itemize}
  and we'll define two more compact names here to be used in an informative 
  example at the end of the post: 
  \begin{itemize}
    \item \(4^{+} = 5\)
    \item \(5^{+} = 6\)
  \end{itemize}
  \label{example:successor}
\end{example}

\subsection{Peano Axioms}

While we prove and refer to the following as theorems, it is common nomenclature 
to refer to these as \textit{The Five Peano Axioms}.

\begin{theorem}
  \( 1 \in \mathbb{N} \)
  \label{thm:first_peano_axiom}
\end{theorem}

\begin{proof}
  Since \( 1 \) is in every inductive set, then \( 1 \in \mathbb{N} \).
\end{proof}

\begin{theorem}
  If \( n \in \mathbb{N} \), then \( n^{+} \in \mathbb{N} \).
  \label{thm:second_peano_axiom}
\end{theorem}

\begin{proof}
  Let \( n \in \mathbb{N} \). By definition 
  \( \forall I ( I \text{ is inductive} \Rightarrow n \in I ) \), and so
  for any inductive set \(K\) we can conclude that \( n^{+} \in K \) since
  \( n \in K \). 
  Recall that \( n \in I \) from \autoref{def:natural_numbers}. By defintion, 
  \( n^{+} \in I \). So we conclude from these two facts that 
  \( n^{+} \in \mathbb{N} \).
\end{proof}

\newpage

\begin{theorem}[Principle of Mathematical Induction]
  If \( S \) is an inductive subset of \( \mathbb{N} \), then 
  \( S = \mathbb{N} \).
  \label{thm:third_peano_axiom}
\end{theorem}

\begin{proof}
  We will argue by subset inclusions that \( S = \mathbb{N} \). By hypothesis, 
  we are given that \( S \subseteq \mathbb{N} \). 
  Now suppose that \( n \in \mathbb{N} \). By hypothesis, \( S \) is an 
  inductive set and so by definition of \( \mathbb{N} \), \( n \in S \). 
  Therefore, we conclude that \( S = \mathbb{N} \).
\end{proof}

\begin{definition}
  We refer to invoking \autoref{thm:third_peano_axiom} as an \textit{argument
  by induction}.
\end{definition}

\begin{theorem}
  For all \( n \in \mathbb{N} \), we have that \( n^{+} \neq 1 \). 
  \label{thm:fourth_peano_axiom}
\end{theorem}

\begin{proof}
  Let \( n \in \mathbb{N} \). Recall that \( n^{+} = n \cup \{ n \} \) and so 
  \( n \in n^{+} \). Well \( n \not\in \emptyset \) and so it cannot be the 
  case that \( n^{+} = 1 \). Therefore \( n^{+} \neq 1 \).
\end{proof}


% \begin{axiom}[Regularity]
%   \(
%     \forall x (
%       x \neq \emptyset 
%       \Rightarrow 
%       \exists y ( 
%         y \in x \land (y \cap x \neq \emptyset )
%       )
%     ) 
%   \)
%   \label{ax:regularity}
% \end{axiom}
% 
% \begin{lemma}
%   \( \forall x (x \not\in x ) \)
%   \label{lemma:x_not_in_x}
% \end{lemma}
% 
% \begin{proof}
%   Suppose that \( x \in x \) for some set \( x \). Then for the singleton 
%   \( \{ x \} \), \autoref{ax:regularity} tells us that \( \{ x \} \) contains 
%   an element that is disjoint from \( \{x\} \). But \( x \in x \cap \{x\} \) 
%   and since no other elements are in \( \{x\} \), we arrive at a contradiction.
% \end{proof}

\begin{definition}
  A set \( T \) is transitive if and only if 
  \( x \in T \Rightarrow x \subseteq T \). 
\end{definition}

\begin{lemma}
  Every natural number is transitive.
  \label{lemma:nat_transitive}
\end{lemma}

\begin{proof}
  We argue by induction. 
  Let \( S \) be the set of all transitive natural numbers. \( 1 \in S \) 
  trivially. Suppose \( n \in S \). Since \( n^{+} = n \cup \{n\} \), then 
  for all \( x \in n^{+} \) either \( x \in n \) or \( x = n \). If 
  \( x \in n \), then 
  by our inductive hypothesis that \( n \in S \), we know that 
  \( x \subseteq n\) which allows us to conclude that \( x \subseteq n^{+} \). 
  If \( x = n \), then we also conclude that \( x \subseteq n^{+} \). 
  Therefore, \( n^{+} \in S \) which completes the inductive argument.
\end{proof}

\begin{theorem}
  If \( n, m \in \mathbb{N}\) and \( n^{+} = m^{+} \), then \( n = m \).
  \label{thm:fifth_peano_axiom}
\end{theorem}

\begin{proof}
  Suppose the hypothesis is true. Well then 
  \( n \in n^{+} \Rightarrow n \in m^{+} \) and so either \( n \in m \) or 
  \( n = m \). By the same type of observation, either \( m \in n \) or \( m = n \). 
  If \( n \neq m\), then \( m \in n \) and \( n \in m \). 
  Well then \autoref{lemma:nat_transitive} tells us that 
  \( m \subseteq n \) and \( n \subseteq m \) and so \( n = m \), a 
  contradiction. Therefore it must be the case that \( n = m \).
\end{proof}

\section{Existence and Uniqueness}

\begin{theorem}[Recursion]
  For any set \( X \), if \( a \in X \) and \( f : X \rightarrow X \), then 
  there exists a function \( u: \mathbb{N} \rightarrow X \) such that 
  \( u(1) = a \) and \( u(n^{+}) = f(u(n)) \) for all other natural numbers 
  \( n \). 
  \label{recursion}
\end{theorem}

\begin{proof}
Let \( \mathcal{C} \subseteq \mathcal{P} (\mathbb{N} \times X )\) be defined 
such that \( (1, a) \in c \) and \( (n^{+}, f(x)) \in c \) whenever 
\( ( n, x ) \in c \) for all \( c \in \mathcal{C} \). It's clear to see that 
\( \mathcal{C} \) is nonempty since \( \mathbb{N} \times X \in C \) and so we 
can form the intersection of all sets in \( \mathcal{C} \) which we'll call 
\( u \). 

Let \( S \) be the set of all natural numbers such that if 
\( n \in S \) then there exists \( (n, x) \in u \) and if 
\( (n, w), (n,y) \in u\) then \( w = y \). We will prove inductively that 
\( S = \mathbb{N} \) which establishes that \( u \) is a function. Furthermore, 
given how \( u \) is constructed, such a proof gives us exactly the kind of 
function we are looking for. 

Suppose that \( (1, b) \in u \), \( a \neq b\), and consider the set 
\( M = u \setminus \{ (1,b) \} \).  We claim that \( (n^{+}, f(x)) \in M \) 
whenever \( ( n, x ) \in M \). Since \( (1, a) \in M \), this will give us that 
\( M \in \mathcal{C} \) and 
\( u \not\subseteq M \) which is a contradiction to \( u \) being the 
intersection of all members of \( u \). That will then allow us to say that 
\( (1,a) \in u \) and for all \( (1,b) \in u \) it would be the case that 
\( a = b \). 

So suppose that \( (n^{+}, f(x)) \) is not present in \( M \) for some 
\( (n,x) \in M \). Well by the definition of \( M \), for all \( \alpha \) 
we have that 
\( (\alpha \in u \land \alpha \neq (1,b)) \rightarrow \alpha \in M \) which 
implies by contraposition that if \( \alpha \not\in M \), then 
\( \alpha \not\in u \lor \alpha = (1, b) \). Well  it can't be the case that 
\( (n^{+},f(x)) = (1,b) \) by \autoref{thm:fourth_peano_axiom}. So then we must 
conclude that \( (n^{+}, f(x)) \not\in u \) which 
contradicts \( u \)'s construction since \( (n,x) \in u \) by the 
assumption that \( (n,x) \in M \). We now must conclude that  
\( M \in \mathcal{C} \) which gives us the base case for our inductive argument
as discussed in the previous paragraph.

Now suppose that \( n \in S \) which implies that there is an \( (n,x) \in u \) 
for at most one \( x \). It follows from the definition of \( u \) that 
\( (n^{+}, f(x)) \in u \). Now if \( n^{+} \) isn't in \( S \), then there 
exists \( (n^{+}, y) \in u \) such that \( f(x) \neq y \). Suppose that this 
is the case and consider the set 
\( J = u \setminus \{(n^{+},y)\} \) with some element \( (m, t) \in J \). 

If \( m = n \), then \( t = x \) since \( n \in S \) and so 
\( (n^{+}, f(x)) = (m^{+},f(t)) \in J \). By construction of J, we note that 
\( \alpha \not\in J \rightarrow (\alpha \not\in u \lor \alpha = (n^{+},y))\) 
for all \( \alpha \). 
So if \( m \neq n \), then we know that \( m^{+} \neq n^{+} \) by  
\autoref{thm:fifth_peano_axiom} which means that if 
\( (m^{+}, f(t)) \not\in J \) 
then we must conclude that \( (m^{+}, f(t)) \not\in u \) which is a 
contradiction to the construction of \( u \) and so \( (m^{+}, f(t)) \in J \). 

Therefore in every case, \( (m^{+},f(t)) \in J \)
whenever \((m,t) \in J \). Since it also the case that \((1,a) \in J \) since 
\( n^{+} \neq 1 \) by \autoref{thm:fourth_peano_axiom}, we conclude that 
\(J \in \mathcal{C} \) and \( u \not\subseteq J \) which is a contradiction to 
\( u \)'s construction as being the intersection of all members of 
\( \mathcal{C} \). 

So \( n^{+} \in S \) and by the principal of mathematical induction we conclude 
that \( \mathbb{N}  = S \). 
\end{proof}

\begin{corollary}
  The function defined in \autoref{recursion} is unique.
  \label{thm:recursion_uniqueness}
\end{corollary}

\begin{proof}
  Let \( a \in X \) and \( f: X \rightarrow X \) for some set \( X \). Suppose 
  that there are two functions 
  \( F: \mathbb{N} \rightarrow \mathbb{X}  \) and 
  \( G: \mathbb{N} \rightarrow \mathbb{X} \) given by \autoref{recursion} \\
  where \( F(1) = G(1) = a \) and for all other natural numbers \( n \) we have
  that 
  \( F(n^{+}) = f(F(n)) \) and \( G(n^+) = f(G(n)) \). 
  Let \( F(1) = G(1) \) be the base case for an inductive 
  proof and suppose that \( F(n) = G(n) \) for some natural number \( n \). 
  Well then \( F(n^{+}) = f(F(n)) = f(G(n)) = G(n^{+}) \) by the inductive 
  hypothesis. So by the principle 
  of mathematical induction, \( F \) and \( G \) are the same function.
\end{proof}

\newpage

\section{Examples}

Let's define a couple of familiar concepts using what we've shown.

\begin{definition}[Addition]
  For any natural number \( n \in \mathbb{N}\), we have 
  \( n^{+} \in \mathbb{N}\) by \autoref{thm:second_peano_axiom}. If we apply 
  \( n^{+} \) and the successor function on \( \mathbb{N} \) to 
  \autoref{recursion} and 
  \autoref{thm:recursion_uniqueness}, we can define a unique function 
  \( s_n : \mathbb{N} \rightarrow \mathbb{N} \) where
  \( s_n(1) = n^{+} \) and \(s_n(m^{+}) = s_n(m)^{+}\) for all 
  \( m \in \mathbb{N} \). For any such function \( s_n \) and 
  \( m \in \mathbb{N} \), we refer to \( s_n(m) \) as \( n + m \) and call it 
  the \textit{sum of} \( n \) \textit{and} \( m \). 
  \label{definition:addition_of_naturals}
\end{definition}

\begin{example}
  Recall \autoref{example:successor}. We can see that 
  \[ 2 + 2 = s_2(2) = s_2(1^{+}) = s_2(1)^{+} = (2^{+})^{+} = 3^{+} = 4 \]
  \label{example:two_plus_two}
\end{example}

\begin{definition}[Multiplication]
  For every \( n \in \mathbb{N} \), we take 
  \(s_n: \mathbb{N} \rightarrow \mathbb{N} \) from 
  \autoref{definition:addition_of_naturals} and define 
  \( p_n: \mathbb{N} \rightarrow \mathbb{N} \) where 
  \( p_n(1) = n \) and \\ \( p_n(m^{+}) = s_n(p_n(m)) = n + p_n(m) \). 
  We refer to \( p_n(m) \) as \( n \cdot m \). We call 
  this the \textit{product of} \( n \) \textit{and} \(m\) or \( n \) 
  \textit{multiplied} by \( m \). 
\end{definition}

\begin{example}
  Recall \autoref{example:successor} once more. We can see that 
  \begin{align*}
    2 \cdot 3 &= 2 + (2 \cdot 2) \\ 
              &= 2 + (2 + (2 \cdot 1)) \\ 
              &= 2 + (2 + 2) \\ 
              &= 2 + 4 \text{ by \autoref{example:two_plus_two}} \\ 
              &= (2 + 3)^{+} \\ 
              &= ((2 + 2)^{+})^{+} \\
              &= (4^{+})^{+} \text{ by \autoref{example:two_plus_two}} \\
              &= 6
  \end{align*}
\end{example}

\bibliographystyle{plain}
\bibliography{citations}

\end{document}
