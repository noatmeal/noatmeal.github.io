\documentclass{article}

\usepackage{xcolor}

% To adjust things like the margin.
\usepackage[margin=1in]{geometry}

\usepackage{amsmath}
\usepackage{amssymb}

% For theorems, definitions, etc.
\usepackage{amsthm}

% For math symbols
\usepackage{amsmath}

% For links to use while reading the pdf on a computer. Also provides nice colors
% for links.
\usepackage[unicode]{hyperref}
\hypersetup{
    colorlinks=true,
    linkcolor=blue
}

\theoremstyle{definition}
\newtheorem{definition}{Definition}[section]
% For autoref so that you don't have to put "thing \ref{label}" every time.
% Just put "\autoref{label}" and it will put the right thing.
\newcommand*{\definitionautorefname}{Definition}

\theoremstyle{definition}
\newtheorem{example}{Example}[section]
\newcommand*{\exampleautorefname}{Example}

\theoremstyle{plain}
\newtheorem{aRule}{Rule}[section]
\newcommand*{\aRuleautorefname}{Rule}

\theoremstyle{remark}
\newtheorem{convention}{Convention}[section]
\newcommand*{\conventionautorefname}{Convention}

\theoremstyle{plain}
\newtheorem{theorem}{Theorem}[section]
% Doesn't need an autorefname decleration because it's already been made.

\theoremstyle{remark}
\newtheorem{remark}{Remark}[section]
\newcommand*{\remarkautorefname}{Remark}

\theoremstyle{plain}
\newtheorem{lemma}{Lemma}[section]
\newcommand*{\lemmaautorefname}{Lemma}

\theoremstyle{plain}
\newtheorem{corollary}{Corollary}[section]
\newcommand*{\corollaryautorefname}{Corollary}

\theoremstyle{plain}
\newtheorem{axiom}{Axiom}[section]
\newcommand*{\axiomautorefname}{Axiom}

\title{The Recursion Theorem}
\author{Noatmeal}

\begin{document}

\maketitle

\section{Preliminaries}

\begin{definition}
  \( \mathbb{N} = \{1, 2, ... \} \) and \( n^{+} \) is the image of 
  \( n \in \mathbb{N} \) under the successor function.
\end{definition}

\section{Statement and Theorem}

\begin{theorem}
  For any set \( X \), if \( a \in X \) and \( f : X \rightarrow X \), then 
  there exists a function \( u: \mathbb{N} \rightarrow X \) such that 
  \( u(1) = a \) and \( u(n^{+}) = f(u(n)) \) for all other natural numbers 
  \( n \). 
  \label{recursion}
\end{theorem}

\begin{proof}
Let \( \mathcal{C} \subseteq \mathcal{P} (\mathbb{N} \times X )\) be defined 
such that \( (1, a) \in c \) and \( (n^{+}, f(x)) \in c \) whenever 
\( ( n, x ) \in c \) for all \( c \in \mathcal{C} \). It's clear to see that 
\( \mathcal{C} \) is nonempty since \( \mathbb{N} \times X \in C \) and so we 
can form the intersection of all sets in \( \mathcal{C} \) which we'll call 
\( u \). Let \( S \) be the set of all natural numbers such that if 
\( n \in S \) then there exists \( (n, x) \in u \) and if 
\( (n, w), (n,y) \in u\) then \( w = y \). We will prove inductively that 
\( S = \mathbb{N} \) which establishes \( u \) as a function. Furthermore, 
given how \( u \) is constructed, such a proof gives us exactly the kind of 
function we are looking for. 

Suppose that \( (1, b) \in u \), \( a \neq b\), and consider the set 
\( M = u \setminus \{ (1,b) \} \).  We claim that \( (n^{+}, f(x)) \in M \) 
whenever \( ( n, x ) \in M \). If that weren't the case then some 
\( (n^{+}, f(x)) \) would not be present in \( M \) for some \( (n,x) \in M \). 
Well by the definition of \( M \), for all \( \alpha \) we have that 
\( (\alpha \in u \land \alpha \neq (1,b)) \rightarrow \alpha \in M \) which 
implies that if \( \alpha \not\in M \), then 
\( \alpha \not\in u \lor \alpha = (1, b) \). Well  it can't be the case that 
\( (1,b) = (n^{+},f(x)) \) since \( n^{+} \neq 1 \) for any natural number 
\( n \). So then we must conclude that \( (n^{+}, f(x)) \not\in u \) which 
contradicts \( u \)'s initial construction since \( (n,x) \in u \) by the 
assumption that \( (n,x) \in M \). Therefore, since \((1,a) \in M \) and for any 
other \( (1,b) \in M \) we know that \( a = b \), we conclude that 
\( 1 \in S \). 

Now suppose that \( n \in S \) which implies that there is an \( (n,x) \in u \) 
for at most one \( x \). It follows from the definition of \( u \) that 
\( (n^{+}, f(x)) \in u \). Now if \( n^{+} \) isn't in \( S \), then there 
exists \( (n^{+}, y) \in u \) such that \( f(x) \neq y \). Consider the set 
\( J = u \setminus \{(n^{+},y)\} \) and some \( (m, t) \in J \). If \( m = n \), 
then \( t = x \) since \( n \in S \) and so 
\( (n^{+}, f(x)) = (m^{+},f(t)) \in J \). By construction of J, we note that 
\( \alpha \not\in J \rightarrow (\alpha \not\in u \lor \alpha = (n^{+},y))\). 
So if \( m \neq n \), then we know that \( m^{+} \neq n^{+} \) by the fourth 
Peano axiom and so if \( (m^{+}, f(t)) \not\in J \) then we must conclude that 
\( (m^{+}, f(t)) \not\in u \) which is a contradiction to the construction of 
\( u \). So \( n^{+} \in S \) and by the principal of mathematical induction we 
conclude that \( \mathbb{N}  = S \). 
\end{proof}

\begin{corollary}
  The function defined in \autoref{recursion} is unique.
\end{corollary}

\begin{proof}
  Let \( a \in X \) and \( f: X \rightarrow X \) for some set \( X \). Suppose 
  that there are two functions 
  \( F: \mathbb{N} \rightarrow \mathbb{X}  \) and 
  \( G: \mathbb{N} \rightarrow \mathbb{X} \) where \( F(1) = G(1) = a \), 
  \( F(n^{+}) = f(F(n)) \), and \( G(n^+) = f(G(n)) \) for all natural 
  numbers \( n \). Let \( F(1) = G(1) \) be the base case for an inductive 
  proof and suppose \( F(n) = G(n) \) for some natural number \( n \). Well 
  then \( F(n^{+}) = f(F(n)) = f(G(n)) = G(n^{+}) \). So by the principle of 
  mathematical induction, \( F \) and \( G \) are the same function.
\end{proof}

\end{document}
