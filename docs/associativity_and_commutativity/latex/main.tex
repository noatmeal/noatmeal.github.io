\documentclass{article}

\usepackage[title]{appendix}

\usepackage{url}

\usepackage{amsmath}
\usepackage{amssymb}

% For theorems, definitions, etc.
\usepackage{amsthm}

% For math symbols
\usepackage{amsmath}

\theoremstyle{definition}
\newtheorem{definition}{Definition}[section]
% For autoref so that you don't have to put "thing \ref{label}" every time.
% Just put "\autoref{label}" and it will put the right thing.
\newcommand*{\definitionautorefname}{Definition}

\theoremstyle{definition}
\newtheorem{example}{Example}[section]
\newcommand*{\exampleautorefname}{Example}

\theoremstyle{plain}
\newtheorem{aRule}{Rule}[section]
\newcommand*{\aRuleautorefname}{Rule}

\theoremstyle{remark}
\newtheorem{convention}{Convention}[section]
\newcommand*{\conventionautorefname}{Convention}

\theoremstyle{plain}
\newtheorem{theorem}{Theorem}[section]
% Doesn't need an autorefname decleration because it's already been made.

\theoremstyle{remark}
\newtheorem{remark}{Remark}[section]
\newcommand*{\remarkautorefname}{Remark}

\theoremstyle{plain}
\newtheorem{lemma}{Lemma}[section]
\newcommand*{\lemmaautorefname}{Lemma}

\theoremstyle{plain}
\newtheorem{proposition}{Proposition}[section]
\newcommand*{\propositionautorefname}{Proposition}

\theoremstyle{plain}
\newtheorem{axiom}{Axiom}[section]
\newcommand*{\axiomautorefname}{Axiom}

\theoremstyle{plain}
\newtheorem{corollary}{Corollary}[section]
\newcommand*{\corollaryautorefname}{Corollary}

\AtBeginDocument{\def\chapterautorefname{Chapter}}%
\AtBeginDocument{\def\sectionautorefname{Section}}%
\AtBeginDocument{\def\subsectionautorefname{Section}}%
\AtBeginDocument{\def\subsubsectionautorefname{Section}}%
\AtBeginDocument{\def\paragraphautorefname{Paragraph}}%


% For links to use while reading the pdf on a computer. Also provides nice 
% colors for links.
\usepackage[unicode]{hyperref}
\hypersetup{
    colorlinks=true,
    linkcolor=blue
}
% Puts you at top of figures in hyperlinks
\usepackage[all]{hypcap}

\title{Distributivity, Associativity, and Commutativity}
\author{\url{https://noatmeal.github.io/}}
\date{}

\begin{document}

\maketitle

In \cite{recursion_theorem}, we created some fundamental tools for working with 
the natural numbers. In this post, we'll establish some of the core ideas that 
make working with addition and multiplication (as well as many other types of 
algebraic operations outside of the purview of this post) such a pleasure.

\section{Preliminaries}

We start by recalling the theorems from \cite{recursion_theorem} that restated 
the definitions of addition and multiplication in algebraic terms where 
\( n^{+} \) is the image of \( n \in \mathbb{N} \) under the successor function.   

\begin{theorem}
  \( n + 1 = n^{+} \) where \( n^{+} \) is the image of \( n \in \mathbb{N} \)
  under the successor function.
  \label{thm:n_plus_one}
\end{theorem}

\begin{theorem}
  \( n + (m + 1) = (n + m) + 1 \)
  \label{thm:associativity_base_case}
\end{theorem}

\begin{lemma}
  \( n \cdot 1 = n \)
  \label{lemma:n_times_one}
\end{lemma}

\begin{theorem}
  \( n \cdot (m + 1) = n + (n \cdot m) \)
  \label{thm:distributivity_base_case}
\end{theorem}


\subsection{Induction}

It's often easier to work with \autoref{thm:third_peano_axiom} from 
\cite{recursion_theorem} by restating it in an algebraic form. 

\begin{theorem}[Principle of Mathematical Induction]
  If \( S \) is an inductive subset of \( \mathbb{N} \), then 
  \( S = \mathbb{N} \).
  \label{thm:third_peano_axiom}
\end{theorem}

\begin{theorem}
  If \( S \subseteq \mathbb{N} \), \( 1 \in S \), and for all \( n \in S \) it's 
  the case that \( n + 1 \in S \), then \( S = \mathbb{N} \).
  \label{thm:restated_peano_axiom}
\end{theorem}

\begin{proof}
  Considering the equality between \( n + 1 \) and \( n^{+} \) as stated 
  in \autoref{thm:n_plus_one}, we can see that \( S \) is an inductive subset
  of \( \mathbb{N} \) and so the hypothesis of \autoref{thm:third_peano_axiom} 
  is fulfilled and which leads us to conclude that \( S = \mathbb{N} \).
\end{proof}

\newpage

\section{A Few ``Nice to Have'' Theorems}

\subsection{Addition}

\begin{theorem}[Associativity of Addition]
 \( x + ( y + z )  = (x + y) + z \) for all \( x, y, z \in \mathbb{N} \).
\end{theorem}

\begin{proof}
  We proceed by induction on \( z \). Let \( S \) be the set of all natural 
  numbers such that for \( n \in S \) we have \( x + (y + n) = (x + y) + n\). 
  Well by \autoref{thm:associativity_base_case}, we know that 
  \( x + (y + 1) = (x + y) + 1 \) and so \( 1 \in S \). So suppose that 
  \( n \in S \). Using \autoref{thm:associativity_base_case} and our inductive 
  hypothesis that \( n \in S \), we can see that
  \begin{align*}
    x + (y + (n + 1)) &= x + ((y + n) + 1) \\
                      &= (x + (y + n)) + 1 \\
                      &= ((x + y) + n) + 1 \\
                      &= (x + y) + (n + 1)
  \end{align*}
  So by \autoref{thm:restated_peano_axiom}, \( S = \mathbb{N} \).
\end{proof}

\begin{lemma}
  \( 1 + n = n + 1 \)
  \label{thm:commutativity_base_case}
\end{lemma}

\begin{proof}
  We proceed by induction on \( n \) and let \( S \) be the set of all natural 
  numbers such that \( 1 + n = n + 1 \). Since \( 1 + 1 = 1 + 1 \) we can see 
  that \( 1 \in S \). Now let \( n \in S \) and consider \( 1 + (n + 1)\). Well
  by \autoref{thm:associativity_base_case} and our inductive hypothesis, 
  \( 1 + (n + 1) = (1 + n) + 1 = (n + 1) + 1 \) and so we conclude that 
  \( S = \mathbb{N} \).  
\end{proof}

\begin{theorem}[Commutativity of Addition]
  \( x + y = y + x \) for all \( x, y \in \mathbb{N} \)  
\end{theorem}

\begin{proof}
  We proceed by induction on \( y \) and let \( S \) be the set of all natural 
  numbers such that \( x + y = y + x \). Our base case of \( 1 \in S \) is 
  established by \autoref{thm:commutativity_base_case} and so let's assume that 
  \( n \in S \) and consider \( x + (n + 1) \). Well then using 
  \autoref{thm:associativity_base_case}, \autoref{thm:commutativity_base_case}, 
  and our inductive hypothesis, we can see that 
  \begin{align*}
    x + (n + 1) &= (x + n) + 1 \\
                &= (n + x) + 1 \\ 
                &= n + (x + 1) \\
                &= n + (1 + x) \\
                &= (n + 1) + x
  \end{align*}
  and so \( S = \mathbb{N} \).
\end{proof}

\newpage

\subsection{Multiplication}



\bibliographystyle{plain}
\bibliography{citations}

\end{document}
